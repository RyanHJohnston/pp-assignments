\documentclass{article}
\usepackage{amsmath}
\usepackage{algorithm}
\usepackage[noend]{algpseudocode}

\begin{document}

\begin{center}
    \text{Assignment \#2}\\
    \text{Ryan H. Johnston, ide709}\\
    \textit{NOTE: All sources are at the end of the document}
\end{center}

%%% QUESTION 1 %%%
\noindent Question \#1) \\~\\
(10 points) Find one of the indices where maximum value occurs in an array A[1..n] of integers
in O(1) time on a CRCW Common PRAM model. \\~\\

% \begingroup
% \obeylines
% \input{find_max.txt}%
% \endgroup%

% \[
% \text{for $k = 1$, $k{+}{+}$, while $k < i$}
% \]

\noindent(a) Write Pseudocode.\\~\\
\textbf{Solution: } \text{If duplicate maximum values exist, the highest index will be the one returned.}
\begin{algorithm}
    \caption{Finding Index of Maximum Value}\label{euclid}
    \begin{algorithmic}[1]
        \Function{FindMaxValueIndex}{A}
        \State $\textit{n} \gets \text{length of }\textit{A}$

        \For $\textit{ i } \gets 0$ \textbf{to} ${n - 1}$, \textbf{in parallel}
        \State $m[i] \gets TRUE$
        \EndFor

        \For $\textit{ i } \gets 0$ \textbf{to} ${n - 1}$ \textbf{and} $\textit{ j } \gets 0$ \textbf{to} $n - 1$, \textbf{in parallel}
        \If $A[i]$ \textless $A[j]$
        \State $ m[i] \gets FALSE$
        \EndIf
        \EndFor

        \For $\textit{ i } \gets 0$ \textbf{to} ${n - 1}$, \textbf{in parallel}
        \If $m[i] = TRUE$
        \State $\text{max} \gets A[i]$
        \State $\text{max\_index} \gets i$
        \EndIf
        \EndFor

        \Return \text{max\_index}
        \EndFunction
    \end{algorithmic}
\end{algorithm}

\noindent(b) Calculate $T_{p}$, $S_{p}$, $E_{p}$, cost, and work of your algorithm. \\~\\
\textbf{Solution: }
\begin{itemize}
    \item Sequential time taken, $T_{1}$ = $O(n)$, sequential asymptotic runtime
    \item Parallel time taken, $T_{p}$ = $O(1)$, parallel asymptotic runtime
    \item Speedup, $S_{p}$ = ${T_{1} \over T_{p}}$ = $O(n) \over O(1)$ = $O(n) \neq O(1)$
    \item Efficiency, $E_{p}$ = $S_{p} \over p$ = ${O(n) \over O(1)} \over p$
    \item Cost, $p T_{p}$ = $p \cdot O(1)$
    \item Work, $T_{1} = O(n + n^{2} + n) = O(n^{2})$, total operation count across all processes
\end{itemize}

\break

%%% QUESTION 2 %%%
\noindent Question \#2)\\
(15 points) Design an algorithm for multiplying two square matrices of size $n x $ which uses 
$p \leq n^{3}$ processors and achieves the fastest parallel execution time of $O(log n)$. You may 
assume EREW PRAM model.\\~\\

% QUESTION 2A %
\noindent(a) Give major steps in high level description/pseudocode to answer part (b) - that is,
detailed pseudocode is not needed.\\~\\
\textbf{Solution:} I cannot write pseudocode similar to the solution to problem $1a$, simply because I 
couldn't design one and prove its correctness. Instead, there exists an algorithm called the 
DNS algorithm (see the textbook at Ch. 8.2.3) that much smarter people designed and published. The algorithm 
can multiply two $n$ x $n$ matrices up to $n^{3}$ processes and performs multiplication in time of 
$O(log n)$ by using $\Omega(n3) \over \log (n)$ processes. The following is roughly-translated pseudocode for how 
the algorithm is done.\\~\\
\begin{enumerate}
    \item Arrange $n^{3}$ processes in a three-dimensional $n$ x $n$ x $n$ logical array.
    \item Each process is labeled according to their location in the array.
    \item Each labeled process performs a single multiplication that are assigned to a single working process.
        Mathematically, it can be written as 
        \begin{equation}
            P[i,j,k] = A_{i}{j} \cdot B_{k}{j}, (0 \leq i,j,k \textless n)
        \end{equation}
    \item Once each process performs a single multplication, as per the previous step, then the contents of 
        \[ C_{i}{j} = \sum_{k=0}^{n-1} P[i,j,k] \]
        Each element $C_{i}{j}$ of the product matrix is obtained by an all-to-one reduction along the $k$ axis,
        therefore giving the resulting matrix.
\end{enumerate}

% QUESTION 2B %
\noindent(b) For $p \leq n^{3}$ processors, calculate expressions for $T_{p}, S_{p}, E_{p}$, 
cost and work of your algorithm as functions of $n$ and $p$ using $O$ notation.\\~\\
\textbf{Solution: }\\~\\
\begin{itemize}
    \item Number of processors, $p$ = ($n$ x $n$) $\cdot$ ($n$ x $n$), $p \leq n^{3}$
    \item Sequential time taken, $T_{1}$ = $O(n^{3})$, sequential asymptotic runtime
    \item Parallel time taken, $T_{p}$ = $O(\log n)$, parallel asymptotic runtime
    \item Speedup, $S_{p}$ = ${T_{1} \over T_{p}}$ = $O(n^{3}) \over O(\log n)$ = $O(n^{3}) \neq O(\log n)$
    \item Efficiency, $E_{p}$ = $S_{p} \over p$ = ${O(n^{3}) \over O(\log n)} \over p$
    \item Cost, $p T_{p}$ = $p \cdot O(\log n)$
    \item Work, $T_{1} = O(n^{3} + n^{3} + n^{3}) = O(n^{3})$, total operation count across all processes
\end{itemize}

\begin{thebibliography}{9}
    \bibitem{texbook}
    Ananth Grama, Anshul Gupta, George Karypls, Vipin Kumar (1994), \emph{Introduction to Parallel Computing, Second Edition},
    The Benjamin/Cummings Publishing Company, Inc. 1994, Pearson Education Limited 2003.

    \bibitem{website} https://www3.nd.edu/~zxu2/acms60212-40212/Lec-07-3.pdf


    % \bibitem{texbook}
    % Donald E. Knuth (1986) \emph{The \TeX{} Book}, Addison-Wesley Professional.

    % \bibitem{lamport94}
    % Leslie Lamport (1994) \emph{\LaTeX: a document preparation system}, Addison
    % Wesley, Massachusetts, 2nd ed.
\end{thebibliography}


\end{document}



\end{}

\end{}
