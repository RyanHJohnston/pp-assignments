\documentclass{article}
\usepackage{amsmath, amssymb, amsthm}
\usepackage{enumitem}


\begin{document}

\begin{center}
    \textbf{Homework 5}
\end{center}

% \noindent \textbf{Definitions}:
% \begin{enumerate}
%     \item A \textit{binary tree} of depth \( k \) has \( 2^{k+1} - 1 \) nodes.
%     \item A \textit{2D-mesh} of size \( k \times k \) has \( k^2 \) nodes.
% \end{enumerate}

% \noindent \textbf{Goal}:
% To prove that the binary tree of depth \( k > 4 \) cannot be embedded on a 2D-mesh with the given properties, we need to show:
% \[ 2^{k+1} - 1 > 2k^2 + 2k + 1 \]

\noindent \underline{ Problem 2.1:}\\
\noindent \textit{Base Case}:
For \( k = 5 \):

\[ \text{LHS: } 2^{5+1} - 1 = 63 \]
\[ \text{RHS: } 2(5^2) + 2(5) + 1 = 61 \]

Clearly, \( 63 > 61 \), so the statement holds true for the base case.

\noindent \textit{Inductive Hypothesis}:
Assume the inequality holds for some arbitrary \( k = n \):
\[ 2^{n+1} - 1 > 2n^2 + 2n + 1 \]

\noindent \textit{Inductive Step}:
For \( k = n + 1 \):

Starting with the LHS:
\[ 2^{n+2} - 1 = 2(2^{n+1}) - 1 \]

Using the inductive hypothesis:
\[ 2^{n+1} > 2n^2 + 2n + 1 \]

We get:
\[ 2(2^{n+1}) - 1 > 2(2n^2 + 2n + 1) - 1 \]
\[ = 4n^2 + 4n \]

For the RHS:
\[ 2(n+1)^2 + 2(n+1) + 1 \]
\[ = 2n^2 + 6n + 5 \]

The inequality:
\[ 2^{n+2} - 1 > 2(n+1)^2 + 2(n+1) + 1 \]
is equivalent to:
\[ 4n^2 + 4n > 2n^2 + 6n + 5 \]
\[ \Rightarrow 2n^2 - 2n > 5 \]

This is true for all \( n > 4 \).

By the principle of induction, the statement \( 2^{k+1} - 1 > 2k^2 + 2k + 1 \) holds for all \( k > 4 \).

Thus, a binary tree of depth \( k > 4 \) cannot be embedded on a 2D-mesh with load-1, dilation-1,
and congestion-free properties.\\~\\


\noindent \underline {Problem 2.2:}\\
\noindent (a)\\
\begin{enumerate}[label=(\roman*)]
    \item For all-to-all broadcast, each processor sends its $m$ words to every other processor.
        This means every processor receives $m$ words from every other processor. Given $P$
        processors, the time is $T_{1} = P \cdot (ts + m \cdot tw)$. We're not multiplying $P-1$
        since every processor sends its data to itself too, in all all-to-all scenario. After
        receiving all the words, local computation to sum up the elements of the array takes $O(m)$
        time. Therefore, $T_{total(i)} = T_{1}$
    \item Accumulation at a single node. Each of the $P$ processors sends its $m$ words to a single
        node: $T_{2} = (P - 1) \cdot (ts + m \cdot tw)$. With one-to-all broadcast by the root, the
        root sends the accumulated results to all processors including itself:
        $T_{3} = P \cdot (ts + m \cdot tw)$. Therefore, $T_{total(ii)}=  T_{2} + T{3}$.
    \item Just like in the all-to-all broadcast, but since we're only adding numbers, every
        processors sends its sum (1 word) to every other processor, and each processor keeps a
        running sum: $T_{4} = P \cdot (ts + tw)$. Therefore, $T_{total(iii)} = T_{4}$.
\end{enumerate}

\noindent (b) For $ts = 100$, $tw = 1$, and $m$ is very large:\\
\begin{enumerate}[label=(\roman*)]
    \item $T_{total(i)} \approx P \cdot (100 + m)$
    \item $T_{total(ii)} \approx P \cdot (100 + m) + P \cdot (100 + m)$
    \item $T_{total(iii)} \approx P \cdot 101$
\end{enumerate}
For very large $m$, $T_{total(iii)}$ is the smallest since it's not dependent on $m$.\\

\noindent (c) For $ts = 100$, $tw = 1$, and $m = 1$.\\
\begin{enumerate}[label=(\roman*)]
    \item $T_{total(i)} \approx P \cdot 101$
    \item $T_{total(ii)} \approx P \cdot 101 + P \cdot 101$
    \item $T_{total(iii)} \approx P \cdot 101$\\
\end{enumerate}
In summary: (a) whichever is better cannot be determined. (b) $T_{total(iii)}$ is better. (c) Either
$T_{total(i)}$ or $T_{total(iii)}$ is better.\\~\\

\end{document}
